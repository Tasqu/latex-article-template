%% This file contains the potential appendices.
% I personally just like to put all the auxiliary
% or non-scientific backmatter here, but this
% varies by journal. Not all journals even include
% all this information, but I like to keep it in
% the draft/preprint as a reminder.


\section{Nomenclature}
\label{app:nomenclature}
% A table or list of the abbreviations and potentially symbols
% used in the manuscript. Exact format and content
% depends on the journal. If you don't want to do this by hand,
% the `glossaries` package might help you.


\section{CRediT authorship}
\label{app:credit}
% The CRediT authorship statement, not required by all journals,
% but often recommended. Personally, I like it.
% All roles included as an example (2025-08-05).

\begin{description}
    \item[John Doe:] Conceptualization, Data curation, Formal analysis, Funding acquisition, Investigation, Methodology, Project administration, Resources, Software, Supervision, Validation, Visualization, Writing --- original draft, Writing --- review \& editing
    \item[Jane Doe:] Writing --- review \& editing
\end{description}


\section{Acknowledgements}
\label{app:acknowledgements}
% Acknowledge sources of funding and any especially useful help
% which didn't merit an author contribution.


\section{Competing interests}
\label{app:interests}
% A competing interests (or conflict of interest) statement
% is often required by journals, where the authors should
% disclose ANY circumstances that might affect their impartial
% and objective judgement of the presented research.
% Exact form depends on the journal.


\section{Data availability}
\label{app:data}
% Some journals want a data availability statement,
% which I think is a good way to promote open data and science.