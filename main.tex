%% Document and package definitions %%
% Try to keep these to a minimum to avoid potential conflicts with journal templates.

\documentclass[]{article}
\usepackage{graphicx}

% Bibliography configuration with BibLaTeX
\usepackage[]{biblatex}
\addbibresource{references.bib}


%% Document metadata and begin document %%
% This information is often presented very differently
% in different journals.

\title{MANUSCRIPT TITLE HERE}
% Quite common to use a main-sub-title structure.
% Should summarize the `Field/Topic', `Scope/Subject', and `Method'.
% "As concise as possible, as detailed as necessary".
% (Kniivilä et al. 2017, p. XXX)

\author{
    John Doe\thanks{Corresponding author: john.doe@email.com} \thanks{Affiliation, address, country},
    Jane Doe\thanks{Affiliation, address, country}
}
\date{\today}

\begin{document}


%% Frontmatter: Title, abstract, keywords %%

\maketitle

\includegraphics[width=\linewidth]{figs/graphical_abstract.pdf}

%% This file contains the highlights of the manuscript.
% Not required by all journals, and the format varies a lot.

\section*{Highlights}
\begin{itemize}
    \item Highlight 1
    \item Highlight 2
    \item Highlight 3
\end{itemize}

\clearpage

\begin{abstract}
    %% This file contains the abstract.
% Just the text, as different journals can have different
% conventions for how their abstract is formatted.
\end{abstract}
%% This file contains the keywords for the article.
% These are often presented in very different ways
% by different journals.

\begin{description}
    \item[Keywords:] keyword1; keyword2; keyword3
\end{description}


%% Main body text contained in the `\tex` subfolder. %%

%% This file contains the introduction sections.
% This is just my preference for the subsections,
% feel free to tweak according to your preference.

% The introduction should consist of the following:
% 1. Create a research space (CARS): What is this research about? Why is it important?
% 2. Identify a research niche: What is known about the subject? What is missing/lacking? How can we address this?
% 3. Contribution: How does this study address the niche? What are the research questions?
% 4. Structure: How is this study structured? How are we going to answer our questions?
% I personally like to include `Structure' immediately after
% CARS, but that's just my preference.

\section{Introduction}
\label{sec:introduction}
% This section is for the primary motivation,
% and explaining how the paper is organized.


\subsection{Background}
\label{sec:background}
% This subsection is for explaining the scientific background.

Example citation \cite{example}


\subsection{Contribution}
\label{sec_contribution}
% This subsection highlights the contribution of this paper
% with respect to the overall motivation and the
% relevant scientific background. 
%% This file contains the materials and methods sections.
% Verty straightforward. Simply explain all materials and
% methods in as much detail as necessary.

% Ideally, this section fulfills the following criteria:
% 1. It's clear how the presented methods aim to address the research questions.
% 2. The presented methods (and their use) are well justified, and can properly address the research questions.
% 3. The methods are presented in enough detail to allow independently recreating the study.
% (Kniivilä et al. 2017, p. xxx)


\section{Materials and methods}
\label{sec:materials_and_methods}
% First provide and overall description and motivation
% behind the chosen data/methodologies, then proceed
% to explain in more detail, potentially in different
% subsections divided based on the topics.


\subsection{Data}
\label{sec:data}
% Example of a possible subsection division between used
% Data and Models.


\subsection{Models}
\label{sec:models}
% Example of a possible subsection division between used
% Data and Models.
%% This file contains the results sections.
% Presents the experimental setup and the results.
% The setup can be separated into a dedicated section
% as well, depending on your preference.

\section{Results}
\label{sec:results}
% This section first explains the computational details of the
% computer setup and performed simulations, and then proceeds
% to present the results. One can highlight points of interests
% in the results, but deeper discussion on their implications
% should be left to the Discussion section, unless combined.
%% This file contains the discussion sections.
% Often ties back to the `Background' and `Contribution' sections.
% Usually includes the following aspects:
% - How well did the study succeed? How was the research process?
% - What problems or limitations were in the data or methods?
% - What did we learn?
% - How can the results be generalized?
% - How can the results be applied?
% - What future work is necessary or possible?

\section{Discussion}
\label{sec:discussion}
% This section is for discussing the meaning, reliability,
% and implications of the results. Depending on the journal
% and one's preference, this can be merged with the results.
% Subsections can be useful for dividing the discussion into
% subtopics, if the results are similarly divided.
%% This file contains the conclusions sections.

\section{Conclusions}
\label{sec:conclusions}
% This section summarises the key results of the paper,
% and draws conclusions from the discussion.

\appendix
%% This file contains the potential appendices.
% I personally just like to put all the auxiliary
% or non-scientific backmatter here, but this
% varies by journal. Not all journals even include
% all this information, but I like to keep it in
% the draft/preprint as a reminder.

\section{Nomenclature}
\label{app:nomenclature}
% A table or list of the abbreviations and potentially symbols
% used in the manuscript. Exact format and content
% depends on the journal. If you don't want to do this by hand,
% the `glossaries` package might help you.


\section{CRediT authorship}
\label{app:credit}
% The CRediT authorship statement, not required by all journals,
% but often recommended. Personally, I like it.

\begin{description}
    \item[John Doe:] Conceptualization, Methodology, etc.
    \item[Jane Doe:] Writing --- Original Draft, etc.
\end{description}


\section{Acknowledgements}
\label{app:acknowledgements}
% Acknowledge sources of funding and any especially useful help
% which didn't merit an author contribution.


\section{Competing interests}
\label{app:interests}
% A competing interests (or conflict of interest) statement
% is often required by journals, where the authors should
% disclose ANY circumstances that might affect their impartial
% and objective judgement of the presented research.
% Exact form depends on the journal.


\section{Data availability}
\label{app:data}
% Some journals want a data availability statement,
% which I think is a good way to promote open data and science.


%% Bibliography from `references.bib` %%

\printbibliography


%% End document

\end{document}


%% What makes a good research paper? (Personal opinions)

% Clarity and conciseness. To me, research papers exist to convey insights as efficiently as possible.
% Text structure helps with this as well, as well-structured text is faster to browse through.
% I also think that a reader should be able to get a rough idea about the work just by skimming through the figures and tables.
% Thus, outlining the paper like a presentation around key figures and diagrams could work.


%% Content/writing checklist: (Kniivilä et al. 2017, p. XXX)

% 1. Is the title appropriate? Does it describe and properly frame the work?
% 2. Does the abstract describe the final work? Does it meet any external requirements?
% (3. Is the table of contents logical and balanced?)
% 4. Does the introduction describe the aims of this work, and its place in the larger scientific context?
% 5. Do the discussion and conclusions tie into the introduction? Do we address the aims and the scientific context?
% 6. Is the scientific/theoretical background specific enough to properly back up the research questions?
% 7. Are all the presented methods properly justified?
% 8. Can the presented methods justifiably answer/address the research questions?
% 9. Are the methods described in enough detail to allow reproducibility?
% 10. Do the presented results and discussion address the research question?


%% Formatting/finalization checklist: (Kniivilä et al. 2017, p. XXX)

% 11. Does each section/subsection only contain what's relevant based on its title?
% 12. Do paragraphs have clear "idea sentences", and enough supporting sentences and meta-textual connectors?
% 13. Are all tables/figures/equations a necessary and natural part of the text? Are they properly addressed in the body text?
% 14. Are citations formatted according to the requirements?
% 15. Does the references list contain all items that are cited?
% 16. Do all references contain enough information to identify and find the sources?
% 17. Are punctuations, citations, references, etc. formatted properly?
% 18. Does the text contain unnecessary phrases or repetition that could be removed?
% 19. Do we meet all formatting requirements? Are we using a correct template?
% 20. Have I asked for feedback on the content, language, and readability? Have I received it and have I given it the proper thought?